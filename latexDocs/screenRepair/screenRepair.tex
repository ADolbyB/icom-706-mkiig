\documentclass[11pt, letterpaper]{article}
\usepackage[margin=1in]{geometry}
% \usepackage[top=1in, bottom=1in, left=0.75in, right=0.75in, paperwidth=8.5in, paperheight=11in]{geometry}
\usepackage{amsfonts, amssymb, amsmath}
\usepackage{tikz, pgfplots}
\usepackage{graphicx} % insert image files
\usepackage{float} % [H] to put images 'HERE'
\usepackage{caption} % library for caption text

\title{Icom IC-706 MkIIG Screen Polarizer Repair}
\author{by Joel Brigida}
\date{} % leave blank for no date or \today for current date.
\pgfplotsset{compat=1.18} % fix pgfplots compatibility error
\captionsetup[figure]{font=footnotesize,labelfont=footnotesize}

\begin{document}
\maketitle
\thispagestyle{empty} % Remove Title Page Number

\hfill \break
\hfill \break
\hfill \break

\begin{figure}[H] % Place image HERE. Also valid: [t] = TOP, [b] = BOTTOM. Note: [H] requires float pkg
    \centering % center the image and scale to 100% of the text width:
    \includegraphics[width=1.0\textwidth]{../../assets/other/ic706mkiig_stock.jpg} % also valid: [scale=x.x], [height=X.Xin], [width=X.Xin]
    \caption{The Icom IC-706 MkIIG. [image credit: rigpix.com]} % Place caption BELOW image.
\end{figure}

\newpage
\setcounter{page}{1} % Make Page #1 after the title page.

\begin{center}
    \textbf{Description of the Problem:}
\end{center}

\hspace{\parindent}This is a known screen problem described as the LCD screen appearance looks very strange and 
hard to read. This is a slightly more difficult repair than repairing the PCB trace. In this case, the LCD Polarizer
attached to the LCD screen gets old and loses all its polarization properties, which results in the following
screenshots.

\newpage
\setcounter{page}{1} % Make Page #1 after the title page.

\begin{center}
    \textbf{Description of the Problem:}
\end{center}

\newpage

\begin{center}
    \textbf{Repair Procedure:}
\end{center}

\end{document}