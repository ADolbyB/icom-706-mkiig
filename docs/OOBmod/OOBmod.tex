\documentclass[12pt,letterpaper]{article}
\usepackage[margin=0.75in]{geometry}    % 3/4 inch margins
\usepackage{graphicx}                   % For images
\usepackage{hyperref}                   % For hyperlinks
\usepackage{caption}                    % Better captions
\usepackage{indentfirst}                % Force Sections 1st paragraph to Indent like the rest.
\usepackage{float}                      % [H] to put images 'HERE', Also valid: [t] = TOP, [b] = BOTTOM.
\usepackage{enumerate}                  % Customize Enumerated Lists
\usepackage{url}                        % used for custom \url{....} methods
\renewcommand{\baselinestretch}{1.1}    % set default paragraph line spacing
\hypersetup{%
    colorlinks=true,                    % colored text rather than box outlined links (cleaner looking)
    urlcolor=blue
}
\let\oldurl\url                         % Makes \url{} automatically use a custom size.
\renewcommand{\url}[1]{{\small\oldurl{#1}}} % custom \small \url size

% ======================= Document Start ===================================
\begin{document}
% ====================== Title Page Start ==================================
\title{Icom IC-706 MkIIG Out of Band MARS Mod}
\author{Joel M. Brigida}
\date{}                                 % leave blank for no date or \today for current date.
\thispagestyle{empty}                   % No page number on title page

\begin{titlepage}                       % Custom titlepage instead of the default \maketitle.
    \centering
    \vspace*{1cm}
    \rule{\textwidth}{1pt}              % Top title rule (bar)

    \vspace{.7\baselineskip}            % Title
    {\huge \textbf{Icom IC-706 MkIIG \\ \vspace*{.5cm}}}
    \LARGE Out of Band MARS Modification% Subtitle
    
    \rule{\textwidth}{1pt}              % Bottom title rule (bar)
    \vspace{1cm}
    \large                              % Set this size for the remaining titlepage.
    Written in \LaTeX\
    \vspace{3cm}                        % More authors can be inserted here with additional minipages.

    \includegraphics[width=.7\textwidth]{../../img/ic706mkiig_lrg.jpg}
    \begin{figure}[H]                   % Place caption HERE. Note: [H] requires float pkg
        \centering                      % center caption
        \caption{The Icom IC-706 MkIIG. [image credit: rigpix.com]} % Place caption BELOW graphics.
    \end{figure}
    \vfill

    Joel Brigida, BSCE \\               % University and date information at bottom of titlepage.
    R \& D, Engineering. \\
    % src: https://www.overleaf.com/latex/templates/sample-report-title-page-for-an-university-assignment-or-project/qwqpmbbmfygn
    \begin{minipage}{.5\textwidth}
        \centering
        {\normalsize \texttt{\href{https://github.com/ADolbyB/}{GitHub}}} \, 
        {\normalsize \texttt{\href{https://www.linkedin.com/in/joelmbrigida/}{LinkedIn}}}
    \end{minipage}

\end{titlepage}

\medskip
\pagenumbering{arabic}                  % 1,2,3 type page numbering.
\setcounter{page}{1}                    % Forces THIS page to be page 1
% === ToC Page  1 ==========================================================
\renewcommand{\contentsname}{Table of Contents} % Custom name for Table of Contents

\begin{center}
    \item \tableofcontents              % Table of Contents appears on page 1
\end{center}

\clearpage                              % Start main content on page 2
% === Page 2 ===============================================================
\begin{center}
    \item \section{Disclaimer}
\end{center}

\bigskip
Please note that none of the information in this manual is meant to be legal advise in any way. The author of
this document does not assume any responsibility for the accuracy of the information in this document. This 
document should be regarded as the author's own opinion, and nothing more for legal reasons even though this is a 
well researched subject in the author's own expertise. Do your own research, make sure you have the proper 
equipment, licenses and know the rules for the appropriate RF bands you are operating on. If any of this goes over 
your head, please stop and have someone with the proper knowledge educate you first.

Do not use this modification for anything else, like illegally transmitting on the GMRS band or the CB band. 
I do not recommended operating there on a Part 97 Radio. The radio's VCO was not designed to operate in these ranges, 
which results in lower power output even if the antenna is tuned for these ranges. For Example: GMRS is FCC Part 95 
and as such must use either dedicated FCC Part 95 radios for GMRS, or FCC Part 90 radios that are bandsplit 
appropriately, like a 400 MHz to 470 MHz or a 450 MHz to 512 MHz band split, either of which will cover the 462 MHz 
and 467MHz GMRS channels and can properly operate there.

This modification is meant for MARS operators, who operate on bands which are adjacent or nearly adjacent to the ham 
bands. A weak justification for this modification is the ``Life or Death'' scenario, where the operation on any 
available frequency is allowed in the case of a human life in danger. This is a weak case because calling for help 
on GMRS or CB from your HAM radio will do no good and is a waste of time. This case is more applicable to a citizen 
picking up a police officer's radio and calling in ``MAYDAY! Officer Down, Shots Fired!!'' which the citizen may 
have just witnessed. Such a transmission on an RF band where one is otherwise not permitted to operate is legally 
authorized in such a scenario only because there is a threat of life or great bodily harm. Call for help, then
acknowledge when proper help is dispatched, and stay off the radio unless called and/or help arrives. Do not
attempt to ``ragchew'' with dispatch.

\clearpage
% === Page 3 ===============================================================
\begin{center}
    \item \section{Description of the Modification}
\end{center}

This modification allows the radio to transmit out-of-band. Such a modification is necessary if, for example, you are 
a member of the Military Auxiliary Radio System (MARS), in which out-of-band operation is permitted legally on an 
FCC Part 97 radio meant for the ham bands. This modification is colloquially referred to as a ``MARS Mod''.

\clearpage                              % Start main content on page 2
% === Page 4 ===============================================================
\begin{center}
    \item \section{Modification Procedure}
\end{center}

\begin{center}
    \item \subsection{Preliminary Procedure}
\end{center}

Before working on the radio, please be sure that it is disconnected from all power sources. You
should have a static mat with wriststrap to guard against ESD (Electrostatic Discharge). Only the radio body needs 
to be worked on for this repair, so the head should be removed for that purpose, as shown in Figure 1.1.

\begin{figure}[H]                       % Place image HERE.
    \centering                          % center the image and scale to 100% of the text width:
    \includegraphics[height=5.0in]{../../img/staticMat.jpg} % also valid: [scale=x.x], [height=X.Xin], [width=X.Xin]
    \caption{Ground the static mat and then ground yourself to the component and/or static mat.} % Place caption BELOW image.
\end{figure}

\clearpage
% === Page 5 ===============================================================
\begin{center}
    \item \subsection{Step One}
\end{center}

Starting this modification requires detaching the remote head, and then removing the top cover 
of the IC-706. The top cover is the cover with the speaker, and please note that the speaker can be left attached 
to the cover during the procedure. There are 5 Phillips-style screws: 3 on the face of the top panel, and one on
each side. The cover then flips up toward the front. Refer to Figure 3.1 for a picture with the cover removed.
The speaker wire lead can be disconnected as shown.

\begin{figure}[H] % Place image HERE. Also valid: [t] = TOP, [b] = BOTTOM. Note: [H] requires `float` pkg
    \centering % center the image and scale to 100% of the text width:
    %\includegraphics[height=4.0in]{../../assets/other/remTopCover.jpg} % also valid: [scale=x.x], [height=X.Xin], [width=X.Xin]
    \caption{This modification requires removal of the top cover [5 screws] as shown.} % Place caption BELOW image.
\end{figure}

\clearpage
% === Page 6 ===============================================================
After the top cover is removed, left of center on the front board is a row of jumper pads. 
Refer to figure 3.2 for the exact location. The only resistor in the row of jumper pads is the one that must be 
removed.

\begin{figure}[H] % Place image HERE. Also valid: [t] = TOP, [b] = BOTTOM. Note: [H] requires `float` pkg
    \centering % center the image and scale to 100% of the text width:
    %\includegraphics[height=5in]{../../assets/OOBmod/beforeMod1.jpg} % also valid: [scale=x.x], [height=X.Xin], [width=X.Xin]
    \caption{Location of the resistor that must be removed for the MARS mod to work.} % Place caption BELOW image.
\end{figure}

\end{document}