\documentclass[12pt,letterpaper]{article}
\usepackage[margin=0.75in]{geometry}    % 3/4 inch margins
\usepackage{graphicx}                   % For images
\usepackage{hyperref}                   % For hyperlinks
\usepackage{caption}                    % Better captions
\usepackage{indentfirst}                % Force Sections 1st paragraph to Indent like the rest.
\usepackage{float}                      % [H] to put images 'HERE', Also valid: [t] = TOP, [b] = BOTTOM.
\usepackage{enumerate}                  % Customize Enumerated Lists
\usepackage{url}                        % used for custom \url{....} methods
\renewcommand{\baselinestretch}{1.1}    % set default paragraph line spacing
\hypersetup{
    colorlinks=true,                    % colored text rather than box outlined links (cleaner looking)
    urlcolor=blue
}
\let\oldurl\url                         % Makes \url{} automatically use a custom size.
\renewcommand{\url}[1]{{\small\oldurl{#1}}} % custom \small \url size

% ======================= Document Start ===================================
\begin{document}
% ====================== Title Page Start ==================================
\title{Icom IC-706 MkIIG Out of Band MARS Mod}
\author{Joel M. Brigida}
\date{}                                 % leave blank for no date or \today for current date.
\thispagestyle{empty}                   % No page number on title page

\begin{titlepage}                       % Custom titlepage instead of the default \maketitle.
    \centering
    \vspace*{1cm}
    \rule{\textwidth}{1pt}              % Top title rule (bar)

    \vspace{.7\baselineskip}            % Title
    {\huge \textbf{Icom IC-706 MkIIG \\ \vspace*{.5cm}}}
    \LARGE Out of Band MARS Modification% Subtitle
    
    \rule{\textwidth}{1pt}              % Bottom title rule (bar)
    \vspace{1cm}
    \large                              % Set this size for the remaining titlepage.
    Written in \LaTeX\
    \vspace{3cm}                        % More authors can be inserted here with additional minipages.

    \includegraphics[width=.7\textwidth]{../../img/ic706mkiig_lrg.jpg}
    \begin{figure}[H]                   % Place caption HERE. Note: [H] requires float pkg
        \centering                      % center caption
        \caption{The Icom IC-706 MkIIG. \href{https://www.rigpix.com/icom/ic706mkiig.htm}{Image Credit}} % Place caption BELOW graphics.
    \end{figure}
    \vfill

    Joel Brigida, BSCE \\               % University and date information at bottom of titlepage.
    R \& D, Engineering. \\
    % src: https://www.overleaf.com/latex/templates/sample-report-title-page-for-an-university-assignment-or-project/qwqpmbbmfygn
    \begin{minipage}{.5\textwidth}
        \centering
        {\normalsize \texttt{\href{https://github.com/ADolbyB/}{GitHub}}} \, 
        {\normalsize \texttt{\href{https://www.linkedin.com/in/joelmbrigida/}{LinkedIn}}}
    \end{minipage}

\end{titlepage}

\medskip
\pagenumbering{arabic}                  % 1,2,3 type page numbering.
\setcounter{page}{1}                    % Forces THIS page to be page 1
% === ToC Page  1 ==========================================================
\renewcommand{\contentsname}{Table of Contents} % Custom name for Table of Contents

\begin{center}
    \item \tableofcontents              % Table of Contents appears on page 1
\end{center}

\clearpage                              % Start main content on page 2
% === Page 2 ===============================================================
\begin{center}
    \item \section{Disclaimer}
\end{center}

\medskip
Please note that none of the information in this manual is meant to be legal advise in any way. The author of
this document does not assume any responsibility for the accuracy of the information in this document. This 
document should be regarded as the author's own opinion, and nothing more for legal reasons even though this is a 
well researched subject in the author's own expertise. Do your own research, make sure you have the proper 
equipment, licenses and know the rules for the appropriate RF bands you are operating on. If any of this goes over 
your head, please stop and have someone with the proper knowledge educate you first.

Do not use this modification for anything else, like illegally transmitting on the GMRS band or the CB band. 
I do not recommended operating there on a Part 97 Radio. The radio's VCO was not designed to operate in these ranges, 
which results in lower power output even if the antenna is tuned for these ranges. For Example: GMRS is FCC Part 95 
and as such must use either dedicated FCC Part 95 radios for GMRS, or FCC Part 90 radios that are bandsplit 
appropriately, like a 400 MHz to 470 MHz or a 450 MHz to 512 MHz band split, either of which will cover the 462 MHz 
and 467MHz GMRS channels and can properly operate there.

This modification is meant for MARS operators, who operate on bands which are adjacent or nearly adjacent to the ham 
bands. On the IC-706 MkIIG, this will also allow for operation in the 60 meter ham band (5.3 MHz), which was allocated 
after the production of the radio. Also note that at the time of this writing, the FCC has now allocated new spectrum
in the 60 meter ham band which was approved for woldwide use at the WRC-15 conference. A link to this new information
is on the \href{https://www.arrl.org/news/fcc-allocates-60-meter-world-wide-amateur-band-approved-at-wrc-15-continues-amateur-use-of-four-addi}
{ARRL site here}.

\begin{center}
    \item \section{Description of the Modification}
\end{center}

This modification allows the radio to transmit out-of-band. Such a modification is necessary if, for example, you are 
a member of the Military Auxiliary Radio System (MARS), in which out-of-band operation is permitted legally on an 
FCC Part 97 radio meant for the ham bands. This modification is colloquially referred to as a ``MARS Mod''. This will
also open up the radio transmitter so it will operate on the 60 meter amateur radio band, which was not allocated
at the time of the IC-706 MkIIG production.

\clearpage
% === Page 3 ===============================================================
\begin{center}
    \item \section{Modification Procedure}
\end{center}

\begin{center}
    \item \subsection{Preliminary Procedure}
\end{center}

\medskip
In order to prevent any Electrostatic Discharge (ESD), an anti-static ESD mat should be used as shown in Figure 3. 
Be sure the mat is grounded properly. An excellent ground source is the ground lug on a household 120VAC NEMA 5-15 
or 5-20 outlet. There are some cheap ground lug outlet adapters as seen
\href{https://www.amazon.com/s?k=Banana+Jack+Outlet+Adapter+Universal+Ground+3+Prong+Outlet+Earth+Connection}
{here [Amazon link]} which convert the ground lug into a banana jack.

\bigskip
\bigskip
\begin{figure}[H]                       % Place image HERE. Note: [H] requires float pkg
    \centering                          % center image, scale to 100% of the text width:
    \includegraphics[height=2.5in]{../../img/ESDGrounding.jpg}
    \caption{An example of an outlet ground lug to banana jack adapter.} % Place caption BELOW image.
\end{figure}

\clearpage
% === Page 4 ===============================================================

Before working on the radio, be sure that it is disconnected from all power sources. Only the radio body needs 
to be worked on for this repair, so the head should be removed for that purpose. Using the ground lug adapter and an 
ESD mat kit, ground the mat to the outlet ground, preferably with the side facing away from the technician. At the 
side facing towards the technician, clip the wrist strap to the other side of the ESD mat and put the wrist strap on 
the non-dominant hand so it does not get in the way. Now the ESD mat and the technician are all grounded and at 
equal potential. If you need to purchase an ESD mat kit, there are some kits 
\href{https://www.amazon.com/s?k=ESD+mat+kit}{here [Amazon Link]}. Note in Figure 3 the alligator clips pull off to 
reveal banana jacks. The top lead needs to be connected to the ground lug adapter, and the bottom lead with the wrist 
strap can be alligator clipped to the mat for a proper connection.

\medskip
\begin{figure}[H]                       % Place image HERE. Note: [H] requires float pkg
    \centering                          % center image, scale to 100% of the text width:
    \includegraphics[width=0.98\textwidth]{../../img/staticMat.jpg}
    \caption{Proper setup of an ESD mat with the radio body.}
\end{figure}

\medskip
If the ESD mat does not have buttons to snap the leads onto (most do), then just use alligator clips to make the 
connection to the mat. This still provides the proper electrical connection needed for equal potential.

\clearpage
% === Page 5 ===============================================================
\begin{center}
    \item \subsection{Step One}
\end{center}

\medskip
Starting this modification requires removing the top cover of the radio. The top cover is the cover with the speaker, 
and the speaker can be left attached to the cover during the procedure. There are 5 Phillips style screws: 
3 in the middle of the top panel, and one on each side at the rear. The cover then flips up toward the front of the
radio. The speaker wire lead can be disconnected as shown by pulling the two connector halves apart with the 
technicians fingers. No pliers or picks are needed here and will only damage the connector. Refer to Figure 4 showing 
the cover removed with the area of interest circled.

\medskip
\begin{figure}[H]                       % Place image HERE.
    \centering                          % center the image
    \includegraphics[width=0.98\textwidth]{../../img/topCover2.jpeg}
    \caption{Removal of the top cover [5 screws] and speaker connection.} % Place caption BELOW image.
\end{figure}

\clearpage
% === Page 6 ===============================================================
\begin{center}
    \item \subsection{Step Two}
\end{center}

Left of center on the front board is a row of jumper pads. Refer to figure 5 for the jumper/resistor location. The 
only resistor in the row of jumper pads is the one that must be removed. In Figure 5, the pad with the resistor is 
circled and already removed. This modification requires proficiency with a 
\href{https://www.amazon.com/s?k=hot+air+rework+station}{hot air rework station}. In the author's 
experience, the ``Yihua'' brand of rework stations has proved very reliable. This also requires use of 
\href{https://www.amazon.com/s?k=smd+tweezers}{SMD tweezers} to remove the surface mount jumper after heated, and a 
\href{https://www.amazon.com/s?k=magnifier+headset}{magnifier headset}. Heat up the jumper with moderate heat,
set the blower to a very low speed so the component does not unsolder then blow across the board and get lost. Care 
must be taken since heating up the jumper with the blower may cause adjacent components to heat up and
melt their solder. This is only a problem if they are moved while the solder is melted. In the case of an 
accidentally moved component, make sure the jumper is already removed, then go back after the solder hardens in a 
minute or so, reheat the component and carefully move it back into place with the tweezers.

\medskip
\begin{figure}[H]                       % Place image HERE.
    \centering                          % center the image
    \includegraphics[width=0.95\textwidth,angle=180]{../../img/IC706-MARS-mod.jpg} % Flip image 180 degrees
    \caption{Location of the resistor that must be removed for the MARS mod to work. \\
    \centering \href{http://www.kk4ice.com/?p=377}{Image Credit: KK4ICE}} % Place caption BELOW image.
\end{figure}

\clearpage
% === Page 7 ===============================================================
\begin{center}
    \item \subsection{Step Three}
\end{center}

\medskip
The modification is now complete. Set aside the removed jumper so it does not fall loose into the radio, and 
reassemble in reverse order. Install the top cover, do not forget to reconnect the speaker connector, then reinstall
the five screws. Now the faceplate can be reinstalled, the radio plugged in, and the radio can be tested on the 60
meter band (5.3 MHz) with a dummy load and a wattmeter as shown below. For effective full scale power output, be sure 
the RF power is set to maximum in the IC-706 MkIIG internal menu. If you are unsure how to do this, the author has 
archived the \textit{IC-706 MkIIG Operating Guide} 
\href{https://github.com/ADolbyB/icom-706-mkiig/blob/main/docs/manuals/IC-706MK2G_OG.pdf}{here on GitHub}. 
Whistle loudly close to the mic with the transmitter keyed in SSB mode, and this should output maximum power, since 
SSB is suppressed carrier.

\medskip
\begin{figure}[H]                       % Place image HERE. Note: [H] requires float pkg
    \centering                          % center image, scale to 100% of the text width:
    \includegraphics[width=0.98\textwidth]{../../img/TestSetup.jpg}
    \caption{A proper test setup using a Bird 43 wattmeter and a 100 watt\\ 
    \centering dummy load in VHF 2m band.}
\end{figure}

\end{document}