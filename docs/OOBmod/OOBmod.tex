\documentclass{report}
\usepackage[margin=0.75in, paperwidth=8.5in, paperheight=11in]{geometry}
%\usepackage[top=0in, bottom=0.75in, left=0.75in, right=0.75in, paperwidth=8.5in, paperheight=11in]{geometry}
\usepackage{amsfonts, amssymb, amsmath}
\usepackage{hyperref} % use hyperlinks
\usepackage{tikz, pgfplots}
\usepackage{graphicx} % insert image files
\usepackage{float} % [H] to put images 'HERE'
\usepackage{caption} % library for caption text

\title{Icom IC-706 MkIIG Out of Band Modification}
\author{by Joel Brigida}
\date{} % leave blank for no date or \today for current date.
\pgfplotsset{compat=1.18} % fix pgfplots compatibility error
\captionsetup[figure]{font=footnotesize,labelfont=footnotesize}

\renewcommand{\contentsname}{Table of Contents:} % Custom name for Table of Contents

\begin{document}

% Custom titlepage instead of the default \maketitle for report class.
\begin{titlepage}
    \centering
    \vspace*{1cm}

    % Title and subtitle are enclosed between two rules.
    \rule{\textwidth}{1pt}

    % Title
    \vspace{.7\baselineskip}
    {\huge \textbf{Icom IC-706 MkIIG \\ \vspace*{.5cm} Out of Band [MARS] Modification}}

    % Subtitle
    \vspace*{.5cm}
    {\LARGE by Joel Brigida}
    
    \rule{\textwidth}{1pt}
    \vspace{1cm}

    % Set this size for the remaining titlepage.
    \large

    % Authors side by side, using two minipages as a trick.
    \begin{minipage}{.5\textwidth}
        \centering
        Joel Brigida \\
        % src: https://www.overleaf.com/latex/templates/sample-report-title-page-for-an-university-assignment-or-project/qwqpmbbmfygn
        {\normalsize \texttt{\href{https://joelbrigida.com}{Website}}} \, {\normalsize \texttt{\href{mailto:joel@joelbrigida.com}{Email}}}
    \end{minipage}

    % More authors can be inserted here with additional minipages.
    \vspace{3cm}

    % Report logo.
    \includegraphics[width=.7\textwidth]{../../assets/other/ic706mkiig_stock.jpg} % Image Credit: rigpix.com
    \vfill

    % University and date information at the bottom of the titlepage.
    Joel Brigida \\
    R \& D, Engineering Dept. \\
    Dark Ridge, LLC \\
    2024-06-30 \\

\end{titlepage}

\tableofcontents
\setcounter{page}{1} % Make Page #1 after the title page.

\begin{center}
    \chapter{\textbf{Preliminary Procedures:}}  % Note "chapter" designation for "report" class doc.
\end{center}

\hspace{\parindent}Before working on the radio, please be sure that it is disconnected from all power sources. You
should have a static mat with wriststrap to guard against ESD (Electrostatic Discharge). Only the radio body needs 
to be worked on for this repair, so the head should be removed for that purpose, as shown in Figure 1.1.

\begin{figure}[H] % Place image HERE. Also valid: [t] = TOP, [b] = BOTTOM. Note: [H] requires float pkg
    \centering % center the image and scale to 100% of the text width:
    \includegraphics[height=5.0in]{../../assets/other/ESDmat.jpg} % also valid: [scale=x.x], [height=X.Xin], [width=X.Xin]
    \caption{Ground the static mat and then ground yourself to the component and/or static mat.} % Place caption BELOW image.
\end{figure}

\begin{center}
    \chapter{\textbf{Description of the Modification:}}
\end{center}

\hspace{\parindent}This modification allows the radio to transmit out-of-band. Such a modification
is necessary if, for example, you are a member of the Military Auxiliary Radio System (MARS), in which out-of-band 
operation is permitted legally on an FCC Part 97 radio meant for the ham bands. This modification is colloquially 
referred to as a ``MARS Mod''.

\hspace{\parindent}DISCLAIMER: Please do not use this modification for anything else, like illegally transmitting 
on the GMRS band or the CB band. I do not recommended this on a Part 97 Radio. For Example: GMRS is FCC Part 95 
and as such must use either dedicated FCC Part 95 radios for GMRS, or FCC Part 90 radios that are bandsplit 
appropriately, like a 400 to 470MHz or a 450MHz to 512MHz band split, either of which will cover the 462MHz and 
467MHz GMRS channels. Another (weak) justification for this modification is the ``Life or Death'' scenario, where 
the operation on any available frequency is allowed in the case of human life in danger. This is a weak case 
because it is more applicable to a citizen picking up a police officer's radio and calling in an ``Officer 
Down, Shots Fired!!'' which they may have just witnessed. Such a transmission where one is otherwise not 
permitted to operate is legally authorized in such a scenario. Alternatively, calling for help on GMRS or CB from 
your HAM radio will do you no good!!

\hspace{\parindent}Another DISCLAIMER: I am not a lawyer and the above is not intended to be legal advice in any 
capacity. Do your own research and make sure you have the proper equipment, licenses and know the rules for the 
appropriate RF bands you are operating on. If any of this goes over your head, please stop and have someone with 
the proper knowledge perform this modification.

\begin{center}
    \chapter{\textbf{Modification Procedure:}}
\end{center}

\hspace{\parindent}Starting this modification requires detaching the remote head, and then removing the top cover 
of the IC-706. The top cover is the cover with the speaker, and please note that the speaker can be left attached 
to the cover during the procedure. There are 5 Phillips-style screws: 3 on the face of the top panel, and one on
each side. The cover then flips up toward the front. Refer to Figure 3.1 for a picture with the cover removed.
The speaker wire lead can be disconnected as shown.

\begin{figure}[H] % Place image HERE. Also valid: [t] = TOP, [b] = BOTTOM. Note: [H] requires `float` pkg
    \centering % center the image and scale to 100% of the text width:
    \includegraphics[height=4.0in]{../../assets/other/remTopCover.jpg} % also valid: [scale=x.x], [height=X.Xin], [width=X.Xin]
    \caption{This modification requires removal of the top cover [5 screws] as shown.} % Place caption BELOW image.
\end{figure}

\newpage

\hspace{\parindent}After the top cover is removed, left of center on the front board is a row of jumper pads. 
Refer to figure 3.2 for the exact location. The only resistor in the row of jumper pads is the one that must be 
removed.

\begin{figure}[H] % Place image HERE. Also valid: [t] = TOP, [b] = BOTTOM. Note: [H] requires `float` pkg
    \centering % center the image and scale to 100% of the text width:
    \includegraphics[height=5in]{../../assets/OOBmod/beforeMod1.jpg} % also valid: [scale=x.x], [height=X.Xin], [width=X.Xin]
    \caption{Location of the resistor that must be removed for the MARS mod to work.} % Place caption BELOW image.
\end{figure}

\end{document}