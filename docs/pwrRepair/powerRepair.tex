\documentclass[12pt,letterpaper]{article}
\usepackage[margin=0.75in]{geometry}    % 3/4 inch margins
\usepackage{graphicx}                   % For images
\usepackage{hyperref}                   % For hyperlinks
\usepackage{caption}                    % Better captions
\usepackage{indentfirst}                % Force Sections 1st paragraph to Indent like the rest.
\usepackage{float}                      % [H] to put images 'HERE', Also valid: [t] = TOP, [b] = BOTTOM.
\usepackage{enumerate}                  % Customize Enumerated Lists
\usepackage{url}                        % used for custom \url{....} methods
\renewcommand{\baselinestretch}{1.1}    % set default paragraph line spacing
\hypersetup{%
    colorlinks=true,                    % colored text rather than box outlined links (cleaner looking)
    urlcolor=blue
}
\let\oldurl\url                         % Makes \url{} automatically use a custom size.
\renewcommand{\url}[1]{{\small\oldurl{#1}}} % custom \small \url size
% ======================= Document Start ===================================
\begin{document}
% ====================== Title Page Start ==================================
\title{Icom IC-706 MkIIG Power Cycling Repair}                         

\medskip
\date{}                                 % leave blank for no date or \today for current date.
\author{Written in \LaTeX\ by Joel M. Brigida} 

\medskip
\maketitle                              % Create Document Title
\thispagestyle{empty}                   % No page number on title page
\pagenumbering{arabic}                  % 1,2,3 type page numbering.

\hfill \break
\hfill \break
\hfill \break
\hfill \break
\hfill \break
\hfill \break

\begin{figure}[H]                       % Place image HERE. Note: [H] requires float pkg
    \centering                          % center image, scale to 100% of the text width:
    \includegraphics[width=1.0\textwidth]{../../img/ic706mkiig_lrg.jpg} 
    % also valid: [scale=x.x], [height=X.Xin], [width=X.Xin]
    \caption{The Icom IC-706 MkIIG. [image credit: rigpix.com]} % Place caption BELOW image.
\end{figure}

\clearpage                              % Go to the next page (which will become page 1)
\setcounter{page}{1}                    % Forces THIS page to be page 1
% === ToC Page  1 ==========================================================

\begin{center}
    \item \tableofcontents                    % Table of Contents appears on page 1
\end{center}

\clearpage                              % Start main content on page 2
% === Page 2 ===============================================================
\begin{center}
    \item \section{Disclaimer}
\end{center}

\bigskip
Please note that none of the information in this manual is meant to be legal advise in any way. The author of
this document does not assume any responsibility for the accuracy of the information in this document. This 
document should be regarded as the author's own opinion, and nothing more for legal reasons even though this is a 
well researched subject in the author's own expertise.

\clearpage
% === Page 3 ===============================================================
\begin{center}
    \item \section{Problem Description}
\end{center}

\medskip
This problem can be described as a random cycling of the radio power, and if severe enough, then the entire 
unit may not turn on. I began to notice this rapid power cycling issue when going over bumps at first, until 
the radio quit entirely and would not power on anymore. This problem is due to a broken trace from what appeared 
to be humidity based moisture ingress under a black rubber pad spacer which was adhered to the PCB on top of 
this trace. Most of the work required for the repair is gaining access to the underside of the top front board. 
Care must be taken to NOT DAMAGE the RF coax under the covers of the radio.

\clearpage
% === Page 4 ===============================================================
\begin{center}
    \item \section{Repair Procedure}
\end{center}

\begin{center}
    \item \subsection{Preliminary Procedure}
\end{center}

\medskip
In order to prevent any Electrostatic Discharge (ESD), an anti-static ESD mat should be used as shown in Figure 5. 
Be sure the mat is grounded properly. An excellent ground source is the ground lug on a household 120VAC NEMA 5-15 
or 5-20 outlet. There are some cheap ground lug outlet adapters as seen
\href{https://www.amazon.com/s?k=Banana+Jack+Outlet+Adapter+Universal+Ground+3+Prong+Outlet+Earth+Connection}
{here [Amazon link]} which convert the ground lug into a banana jack.

\bigskip
\bigskip
\begin{figure}[H]                       % Place image HERE. Note: [H] requires float pkg
    \centering                          % center image, scale to 100% of the text width:
    \includegraphics[height=2.25in]{../../img/ESDGrounding.jpg}
    \caption{An example of an outlet ground lug to banana jack adapter.} % Place caption BELOW image.
\end{figure}

\clearpage
% === Page 5 ===============================================================
\medskip
Using the ground lug adapter and an ESD mat kit, ground the mat to the outlet ground, preferably with the side facing 
away from the technician. At the side facing towards the technician, clip the wrist strap to the other side of the 
ESD mat and put the wrist strap on the non-dominant hand so it does not get in the way. Now the ESD mat and the 
technician are all grounded and at equal potential. If you need to purchase an ESD mat kit, there are some kits 
\href{https://www.amazon.com/s?k=ESD+mat+kit}{here [Amazon Link]}. Note in Figure 3 the alligator clips pull off to 
reveal banana jacks. Detach the head of the radio and place it on the ESD mat. Note that in Figure 5, the top lead 
needs to be connected to the ground lug adapter, and the bottom lead with the wrist strap can be alligator clipped 
to the mat for a proper connection.

\medskip
\begin{figure}[H]                       % Place image HERE. Note: [H] requires float pkg
    \centering                          % center image, scale to 100% of the text width:
    \includegraphics[width=0.9\textwidth]{../../img/staticMat.jpg}
    \caption{Proper setup of an ESD mat with the radio body.}
\end{figure}

\medskip
If the ESD mat does not have buttons to snap the leads onto (most do), then just use alligator clips to make the 
connection to the mat. This still provides the proper electrical connection needed for equal potential.

\clearpage
% === Page 6 ===============================================================
\begin{center}
    \item \subsection{Step One}
\end{center}

\medskip
This is a faily straightforward repair. There is a broken trace on the bottom side of the top front PCB, which 
is under the black rubber spacer.

% Notice that the black spacer pad on the PCB in Figure 2 has some oxidation on it, as well as 
% where the pad contacts the case on the underside of the PCB. Looking a bit closer reveals a trace running directly 
% underneath the black pad, which is shown in Figure 3 below:

% \medskip
% \begin{figure}[H]                       % Place image HERE.
%     \centering                          % center the image.
%     \includegraphics[width=0.9\textwidth]{../../img/}
%     \caption{Accessing the bottom of the top front PCB.} % Place caption BELOW image.
% \end{figure}

% The black pad must be removed to reveal the extent of PCB damage. This is shown in figure 4 
% below. Notice on the left side that removing the black pad results in a thin layer left behind. Next, the board
% must be cleaned and prepped so the broken trace can be soldered back to a closed circuit.

% \medskip
% \begin{figure}[H]                       % Place image HERE.
%     \centering                          % center the image.
%     \includegraphics[width=0.9\textwidth]{../../img/} 
%     \caption{A damaged trace is revealed under the corrosion on the pad.} % Place caption BELOW image.
% \end{figure}

\end{document}