\documentclass[11pt, letterpaper]{article}
\usepackage[margin=1in]{geometry}
% \usepackage[top=1in, bottom=1in, left=0.75in, right=0.75in, paperwidth=8.5in, paperheight=11in]{geometry}
\usepackage{amsfonts, amssymb, amsmath}
\usepackage{tikz, pgfplots}
\usepackage{graphicx} % insert image files
\usepackage{float} % [H] to put images 'HERE'
\usepackage{caption} % library for caption text

\title{Icom IC-706 MkIIG Power Cycling Repair}
\author{by Joel Brigida}
\date{} % leave blank for no date or \today for current date.
\pgfplotsset{compat=1.18} % fix pgfplots compatibility error
\captionsetup[figure]{font=footnotesize,labelfont=footnotesize}

\begin{document}
\maketitle
\thispagestyle{empty} % Remove Title Page Number

\begin{figure}[H] % Place image HERE. Also valid: [t] = TOP, [b] = BOTTOM. Note: [H] requires float pkg
    \centering % center the image and scale to 100% of the text width:
    \includegraphics[width=1.0\textwidth]{../../assets/other/ic706mkiig_stock.jpg} % also valid: [scale=x.x], [height=X.Xin], [width=X.Xin]
    \caption{The Icom IC-706 MkIIG. [image credit: rigpix.com]} % Place caption BELOW image.
\end{figure}

\newpage
\setcounter{page}{1} % Make Page #1 after the title page.

\begin{center}
    \textbf{Description of the Problem:}
\end{center}

\hspace{\parindent}This problem can be described as a random cycling of the radio power, and if severe enough,
then the entire unit may not turn on. I began to notice this rapid power cycling issue when going over bumps
at first, until the radio quit entirely.

\hspace{\parindent}This is a faily straightforward repair. There is a broken trace on the bottom side of the top
front PCB, which is under the black rubber pad pictured in Figure 2. Most of the work required for the repair is 
gaining access to the underside of the top front board. Care must be taken to NOT DAMAGE the RF coax under the
covers of the radio.

\begin{figure}[H] % Place image HERE. Also valid: [t] = TOP, [b] = BOTTOM. Note: [H] requires float pkg
    \centering % center the image and scale to 40% of the text width:
    \includegraphics[height=6.7in]{../../assets/powerRepair/Step1a.jpg} % also valid: [scale=x.x], [height=X.Xin], [width=X.Xin]
    \caption{Accessing the bottom of the top front PCB.} % Place caption BELOW image.
\end{figure}

\hspace{\parindent}Notice that the black spacer pad on the PCB in Figure 2 has some oxidation on it, as well as 
where the pad contacts the case on the underside of the PCB. Looking a bit closer reveals a trace running directly 
underneath the black pad, which is shown in Figure 3 below:

\begin{figure}[H] % Place image HERE. Also valid: [t] = TOP, [b] = BOTTOM. Note: [H] requires float pkg
    \centering % center the image and scale to 100% of the text width:
    \includegraphics[height=7.0in]{../../assets/powerRepair/Step1b.jpg} % also valid: [scale=x.x], [height=X.Xin], [width=X.Xin]
    \caption{This trace runs directly underneath the black pad.} % Place caption BELOW image.
\end{figure}

\newpage
\begin{center}
    \textbf{Repair Procedure:}
\end{center}

\hspace{\parindent}The black pad must be removed to reveal the extent of PCB damage. This is shown in figure 4 
below. Notice on the left side that removing the black pad results in a thin layer left behind. Next, the board
must be cleaned and prepped so the broken trace can be soldered back to a closed circuit. 

\begin{figure}[H] % Place image HERE. Also valid: [t] = TOP, [b] = BOTTOM. Note: [H] requires float pkg
    \centering % center the image and scale to 100% of the text width:
    \includegraphics[height=6.5in]{../../assets/powerRepair/Step2a.jpg} % also valid: [scale=x.x], [height=X.Xin], [width=X.Xin]
    \caption{A damaged trace is revealed under the corrosion on the pad.} % Place caption BELOW image.
\end{figure}

\end{document}