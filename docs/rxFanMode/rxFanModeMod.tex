\documentclass[12pt,letterpaper]{article}
\usepackage[margin=0.75in]{geometry}    % 3/4 inch margins
\usepackage{graphicx}                   % For images
\usepackage{textcomp}                   % for degree symbol: \textdegree
\usepackage{hyperref}                   % For hyperlinks
\usepackage{caption}                    % Better captions
\usepackage{indentfirst}                % Force Sections 1st paragraph to Indent like the rest.
\usepackage{float}                      % [H] to put images 'HERE', Also valid: [t] = TOP, [b] = BOTTOM.
\usepackage{enumerate}                  % Customize Enumerated Lists
\usepackage{url}                        % used for custom \url{....} methods
\renewcommand{\baselinestretch}{1.1}    % set default paragraph line spacing
\hypersetup{%
    colorlinks=true,                    % colored text rather than box outlined links (cleaner looking)
    urlcolor=blue
}
\let\oldurl\url                         % Makes \url{} automatically use a custom size.
\renewcommand{\url}[1]{{\small\oldurl{#1}}} % custom \small \url size

% ======================= Document Start ===================================
\begin{document}

% ====================== Title Page Start ==================================
\title{Icom IC-706 MkIIG RX Fan Mode Modification}
\author{Joel M. Brigida}
\date{}                                 % leave blank for no date or \today for current date.
\thispagestyle{empty}                   % No page number on title page

\begin{titlepage}                       % Custom titlepage instead of the default \maketitle.
    \centering
    \vspace*{1cm}

    \rule{\textwidth}{1pt}              % Top title rule (bar)

    \vspace{.7\baselineskip}            % Title
    {\huge \textbf{Icom IC-706 MkIIG \\ \vspace*{.5cm}}}

    \LARGE RX Fan Mode Modification     % Subtitle
    
    \rule{\textwidth}{1pt}              % Bottom title rule (bar)
    \vspace{1cm}
    \large Written in \LaTeX\           % Set this size for the remaining titlepage.
    
    \vspace{3cm}                        % More authors can be inserted here with additional minipages.

    \includegraphics[width=.7\textwidth]{../../img/ic706mkiig_lrg.jpg}
    \begin{figure}[H]                   % Place caption HERE. Note: [H] requires float pkg
        \centering                      % center caption
        \caption{The Icom IC-706 MkIIG. \href{https://www.rigpix.com/icom/ic706mkiig.htm}{Image Credit}}
    \end{figure}
    \vfill

    Joel Brigida, BSCE \\               % University and date information at bottom of titlepage.
    R \& D, Engineering. \\
    % src: https://www.overleaf.com/latex/templates/sample-report-title-page-for-an-university-assignment-or-project/qwqpmbbmfygn
    \begin{minipage}{.5\textwidth}
        \centering
        {\normalsize \texttt{\href{https://github.com/ADolbyB/}{GitHub}}} \, 
        {\normalsize \texttt{\href{https://www.linkedin.com/in/joelmbrigida/}{LinkedIn}}}
    \end{minipage}

\end{titlepage}

\medskip
\pagenumbering{arabic}                  % 1,2,3 type page numbering.
\setcounter{page}{1}                    % Forces THIS page to be page 1
% === ToC Page  1 ==========================================================
\renewcommand{\contentsname}{Table of Contents} % Custom name for Table of Contents

\begin{center}
    \item \tableofcontents              % Table of Contents appears on page 1
\end{center}

\clearpage                              % Start main content on page 2
% === Page 2 ===============================================================
\begin{center}
    \item \section{Disclaimer}
\end{center}

\bigskip
Please note that none of the information in this manual is meant to be legal advise in any way. The author of
this document does not assume any responsibility for the accuracy of the information in this document. This 
document should be regarded as the author's own opinion, and nothing more for legal reasons even though this is a 
well researched subject in the author's own expertise.

\begin{center}
    \item \section{Problem Description}
\end{center}


\medskip
Unlike the IC-706, the IC-706MkIIG do not keep its internal blower running in receive mode. So, the radio becomes very 
hot after half-hour or more of operation, especially in summer.

A small 200 ohms 1 watt 1\% metal-film resistor connected in the PA board, between the switched +14v and the blower 
positive lead will make it run very quietly, dropping the external body temperature of the MkIIG very near to that 
found in the IC-706. The voltage applied by the resistor over the cooler is 4.7 volts and it is sufficient to start 
and keep it running without any perceptible noise.

Resist to temptation of increase the receive-mode blower speed (reducing the resistor ohmic value) avoiding to 
transform your radio in a vacuum cleaner. After this modification, the measured radio body temperature after two 
hours of receive-only mode was 37 \textdegree C at room temperature of 25 \textdegree C (98.6/77 \textdegree F).

This mod will not alter the transmit behavior of the ventilation system that will remain operating as usual or even 
better because the 46.5 mA auxiliary current supplied by this resistor will keep the blower drive transistor more 
cooler. Another benefit is the reduced frequency drift in the VHF and UHF bands, SSB mode*. One of the resistor 
lead is connected to the positive of the blower connector and the other in the RF choke (L613) loop as show in the 
pictures.

If you want added security, provide some isolation on the resistor leads. DO NOT short circuit the choke winding and 
be very careful when soldering the resistor lead over the male PC connector pin.

In this mod, there is no necessity of board removal and could be done in a matter of minutes. Only the radio bottom 
cover must be removed (5 screws).

Apparently, the original Icom IC-706 was designed so that the fan would blow constantly. When they designed the 
mkII and mkIIg, they added a fan circuit to limit the noise and current drain. Unfortunately, once that circuit 
gets marginal, it stops being useful and actually contributes to the radio's demise.

Among the references is a rework involving the addition of a 200$ \Omega $ 1W resistor between L50 and J2 
(on the IC-706mkII, at least) which will provide a constant voltage to keep the fan moving at a slower speed. 
The benefit is that the fan controller won't need to start the fan, it just ramps up to the right speed.

High temp is known as the worst enemy of ANY electronic component, both digital or analog devices. Now the Icom 
IC-706MKIIG was opened it was time to do something about the high temperature during RX (receive) mode. A 220 ohms 
resistor connected in the PA board, between the switched +13.8 volts and the blower positive lead will make the fan 
run all the time, very quietly and dropping the case temperature to very near that one show by former 706. This mod 
is active only in RX mode. It will not change or disable the TX (transmit) blower rotation that is controlled by the 
temperature sensor. Therefore, in TX mode, the ventilation system will operate as usual.

\clearpage
% === Page xx ===============================================================
\begin{center}
    \item \section{References}
\end{center}

\begin{enumerate}
    \item \href{https://mods-ham.darc.de/02_Mods/Icom-Mods/IC-706MKIIG/IC-706MKIIG_Fan\%20Mode.htm}
        {\textit{iCOM IC-706MkIIG Fan Mode} : mods-ham.darc.de}
    \item \href{https://www.phaysis.com/2018/08/11/icom-ic-706mkii-fan-mod/}
        {\textit{iCOM IC-706MkIIG Fan Mod} : phaysis.com}
    \item \href{https://www.pc2c.nl/icom_706.php}
        {\textit{iCOM IC-706MkIIG Cooling Fan Modification} : pc2c.nl}
    \item \href{https://ik4rvg.altervista.org/alterpages/files/IC-706MKII.pdf}
        {\textit{KB2LJJ Radio Mods Database} : ik4rvg.altervista.org}
\end{enumerate}

\end{document}